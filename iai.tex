
%% bare_conf.tex
%% V1.4b
%% 2015/08/26
%% by Michael Shell
%% See:
%% http://www.michaelshell.org/
%% for current contact information.
%%
%% This is a skeleton file demonstrating the use of IEEEtran.cls
%% (requires IEEEtran.cls version 1.8b or later) with an IEEE
%% conference paper.
%%
%% Support sites:
%% http://www.michaelshell.org/tex/ieeetran/
%% http://www.ctan.org/pkg/ieeetran
%% and
%% http://www.ieee.org/

%%*************************************************************************
%% Legal Notice:
%% This code is offered as-is without any warranty either expressed or
%% implied; without even the implied warranty of MERCHANTABILITY or
%% FITNESS FOR A PARTICULAR PURPOSE! 
%% User assumes all risk.
%% In no event shall the IEEE or any contributor to this code be liable for
%% any damages or losses, including, but not limited to, incidental,
%% consequential, or any other damages, resulting from the use or misuse
%% of any information contained here.
%%
%% All comments are the opinions of their respective authors and are not
%% necessarily endorsed by the IEEE.
%%
%% This work is distributed under the LaTeX Project Public License (LPPL)
%% ( http://www.latex-project.org/ ) version 1.3, and may be freely used,
%% distributed and modified. A copy of the LPPL, version 1.3, is included
%% in the base LaTeX documentation of all distributions of LaTeX released
%% 2003/12/01 or later.
%% Retain all contribution notices and credits.
%% ** Modified files should be clearly indicated as such, including  **
%% ** renaming them and changing author support contact information. **
%%*************************************************************************


% *** Authors should verify (and, if needed, correct) their LaTeX system  ***
% *** with the testflow diagnostic prior to trusting their LaTeX platform ***
% *** with production work. The IEEE's font choices and paper sizes can   ***
% *** trigger bugs that do not appear when using other class files.       ***                          ***
% The testflow support page is at:
% http://www.michaelshell.org/tex/testflow/



\documentclass[conference]{IEEEtran}
\usepackage{amsmath}

% limits underneath
\DeclareMathOperator*{\argmaxA}{arg\,max} % Jan Hlavacek
\DeclareMathOperator*{\argmaxB}{argmax}   % Jan Hlavacek
\DeclareMathOperator*{\argmaxC}{\arg\max}   % rbp

\newcommand{\argmaxD}{\arg\!\max} % AlfC

\newcommand{\argmaxE}{\mathop{\mathrm{argmax}}}          % ASdeL
\newcommand{\argmaxF}{\mathop{\mathrm{argmax}}\limits}   % ASdeL

% limits on side
\DeclareMathOperator{\argmaxG}{arg\,max} % Jan Hlavacek
\DeclareMathOperator{\argmaxH}{argmax}   % Jan Hlavacek
\newcommand{\argmaxI}{\mathop{\mathrm{argmax}}\nolimits} % ASdeL

\newcommand{\cs}[1]{\texttt{\symbol{`\\}#1}}

% Some Computer Society conferences also require the compsoc mode option,
% but others use the standard conference format.
%
% If IEEEtran.cls has not been installed into the LaTeX system files,
% manually specify the path to it like:
% \documentclass[conference]{../sty/IEEEtran}





% Some very useful LaTeX packages include:
% (uncomment the ones you want to load)


% *** MISC UTILITY PACKAGES ***
%
%\usepackage{ifpdf}
% Heiko Oberdiek's ifpdf.sty is very useful if you need conditional
% compilation based on whether the output is pdf or dvi.
% usage:
% \ifpdf
%   % pdf code
% \else
%   % dvi code
% \fi
% The latest version of ifpdf.sty can be obtained from:
% http://www.ctan.org/pkg/ifpdf
% Also, note that IEEEtran.cls V1.7 and later provides a builtin
% \ifCLASSINFOpdf conditional that works the same way.
% When switching from latex to pdflatex and vice-versa, the compiler may
% have to be run twice to clear warning/error messages.






% *** CITATION PACKAGES ***
%
%\usepackage{cite}
% cite.sty was written by Donald Arseneau
% V1.6 and later of IEEEtran pre-defines the format of the cite.sty package
% \cite{} output to follow that of the IEEE. Loading the cite package will
% result in citation numbers being automatically sorted and properly
% "compressed/ranged". e.g., [1], [9], [2], [7], [5], [6] without using
% cite.sty will become [1], [2], [5]--[7], [9] using cite.sty. cite.sty's
% \cite will automatically add leading space, if needed. Use cite.sty's
% noadjust option (cite.sty V3.8 and later) if you want to turn this off
% such as if a citation ever needs to be enclosed in parenthesis.
% cite.sty is already installed on most LaTeX systems. Be sure and use
% version 5.0 (2009-03-20) and later if using hyperref.sty.
% The latest version can be obtained at:
% http://www.ctan.org/pkg/cite
% The documentation is contained in the cite.sty file itself.






% *** GRAPHICS RELATED PACKAGES ***
%
\ifCLASSINFOpdf
% \usepackage[pdftex]{graphicx}
% declare the path(s) where your graphic files are
% \graphicspath{{../pdf/}{../jpeg/}}
% and their extensions so you won't have to specify these with
% every instance of \includegraphics
% \DeclareGraphicsExtensions{.pdf,.jpeg,.png}
\else
% or other class option (dvipsone, dvipdf, if not using dvips). graphicx
% will default to the driver specified in the system graphics.cfg if no
% driver is specified.
% \usepackage[dvips]{graphicx}
% declare the path(s) where your graphic files are
% \graphicspath{{../eps/}}
% and their extensions so you won't have to specify these with
% every instance of \includegraphics
% \DeclareGraphicsExtensions{.eps}
\fi
% graphicx was written by David Carlisle and Sebastian Rahtz. It is
% required if you want graphics, photos, etc. graphicx.sty is already
% installed on most LaTeX systems. The latest version and documentation
% can be obtained at: 
% http://www.ctan.org/pkg/graphicx
% Another good source of documentation is "Using Imported Graphics in
% LaTeX2e" by Keith Reckdahl which can be found at:
% http://www.ctan.org/pkg/epslatex
%
% latex, and pdflatex in dvi mode, support graphics in encapsulated
% postscript (.eps) format. pdflatex in pdf mode supports graphics
% in .pdf, .jpeg, .png and .mps (metapost) formats. Users should ensure
% that all non-photo figures use a vector format (.eps, .pdf, .mps) and
% not a bitmapped formats (.jpeg, .png). The IEEE frowns on bitmapped formats
% which can result in "jaggedy"/blurry rendering of lines and letters as
% well as large increases in file sizes.
%
% You can find documentation about the pdfTeX application at:
% http://www.tug.org/applications/pdftex





% *** MATH PACKAGES ***
%
%\usepackage{amsmath}
% A popular package from the American Mathematical Society that provides
% many useful and powerful commands for dealing with mathematics.
%
% Note that the amsmath package sets \interdisplaylinepenalty to 10000
% thus preventing page breaks from occurring within multiline equations. Use:
%\interdisplaylinepenalty=2500
% after loading amsmath to restore such page breaks as IEEEtran.cls normally
% does. amsmath.sty is already installed on most LaTeX systems. The latest
% version and documentation can be obtained at:
% http://www.ctan.org/pkg/amsmath





% *** SPECIALIZED LIST PACKAGES ***
%
%\usepackage{algorithmic}
% algorithmic.sty was written by Peter Williams and Rogerio Brito.
% This package provides an algorithmic environment fo describing algorithms.
% You can use the algorithmic environment in-text or within a figure
% environment to provide for a floating algorithm. Do NOT use the algorithm
% floating environment provided by algorithm.sty (by the same authors) or
% algorithm2e.sty (by Christophe Fiorio) as the IEEE does not use dedicated
% algorithm float types and packages that provide these will not provide
% correct IEEE style captions. The latest version and documentation of
% algorithmic.sty can be obtained at:
% http://www.ctan.org/pkg/algorithms
% Also of interest may be the (relatively newer and more customizable)
% algorithmicx.sty package by Szasz Janos:
% http://www.ctan.org/pkg/algorithmicx




\usepackage{amsmath}
\DeclareMathOperator*{\argmax}{arg\!\max}
\DeclareMathOperator*{\argmin}{arg\!\min}



% *** ALIGNMENT PACKAGES ***
%
%\usepackage{array}
% Frank Mittelbach's and David Carlisle's array.sty patches and improves
% the standard LaTeX2e array and tabular environments to provide better
% appearance and additional user controls. As the default LaTeX2e table
% generation code is lacking to the point of almost being broken with
% respect to the quality of the end results, all users are strongly
% advised to use an enhanced (at the very least that provided by array.sty)
% set of table tools. array.sty is already installed on most systems. The
% latest version and documentation can be obtained at:
% http://www.ctan.org/pkg/array


% IEEEtran contains the IEEEeqnarray family of commands that can be used to
% generate multiline equations as well as matrices, tables, etc., of high
% quality.




% *** SUBFIGURE PACKAGES ***
%\ifCLASSOPTIONcompsoc
%  \usepackage[caption=false,font=normalsize,labelfont=sf,textfont=sf]{subfig}
%\else
%  \usepackage[caption=false,font=footnotesize]{subfig}
%\fi
% subfig.sty, written by Steven Douglas Cochran, is the modern replacement
% for subfigure.sty, the latter of which is no longer maintained and is
% incompatible with some LaTeX packages including fixltx2e. However,
% subfig.sty requires and automatically loads Axel Sommerfeldt's caption.sty
% which will override IEEEtran.cls' handling of captions and this will result
% in non-IEEE style figure/table captions. To prevent this problem, be sure
% and invoke subfig.sty's "caption=false" package option (available since
% subfig.sty version 1.3, 2005/06/28) as this is will preserve IEEEtran.cls
% handling of captions.
% Note that the Computer Society format requires a larger sans serif font
% than the serif footnote size font used in traditional IEEE formatting
% and thus the need to invoke different subfig.sty package options depending
% on whether compsoc mode has been enabled.
%
% The latest version and documentation of subfig.sty can be obtained at:
% http://www.ctan.org/pkg/subfig




% *** FLOAT PACKAGES ***
%
%\usepackage{fixltx2e}
% fixltx2e, the successor to the earlier fix2col.sty, was written by
% Frank Mittelbach and David Carlisle. This package corrects a few problems
% in the LaTeX2e kernel, the most notable of which is that in current
% LaTeX2e releases, the ordering of single and double column floats is not
% guaranteed to be preserved. Thus, an unpatched LaTeX2e can allow a
% single column figure to be placed prior to an earlier double column
% figure.
% Be aware that LaTeX2e kernels dated 2015 and later have fixltx2e.sty's
% corrections already built into the system in which case a warning will
% be issued if an attempt is made to load fixltx2e.sty as it is no longer
% needed.
% The latest version and documentation can be found at:
% http://www.ctan.org/pkg/fixltx2e


%\usepackage{stfloats}
% stfloats.sty was written by Sigitas Tolusis. This package gives LaTeX2e
% the ability to do double column floats at the bottom of the page as well
% as the top. (e.g., "\begin{figure*}[!b]" is not normally possible in
% LaTeX2e). It also provides a command:
%\fnbelowfloat
% to enable the placement of footnotes below bottom floats (the standard
% LaTeX2e kernel puts them above bottom floats). This is an invasive package
% which rewrites many portions of the LaTeX2e float routines. It may not work
% with other packages that modify the LaTeX2e float routines. The latest
% version and documentation can be obtained at:
% http://www.ctan.org/pkg/stfloats
% Do not use the stfloats baselinefloat ability as the IEEE does not allow
% \baselineskip to stretch. Authors submitting work to the IEEE should note
% that the IEEE rarely uses double column equations and that authors should try
% to avoid such use. Do not be tempted to use the cuted.sty or midfloat.sty
% packages (also by Sigitas Tolusis) as the IEEE does not format its papers in
% such ways.
% Do not attempt to use stfloats with fixltx2e as they are incompatible.
% Instead, use Morten Hogholm'a dblfloatfix which combines the features
% of both fixltx2e and stfloats:
%
% \usepackage{dblfloatfix}
% The latest version can be found at:
% http://www.ctan.org/pkg/dblfloatfix




% *** PDF, URL AND HYPERLINK PACKAGES ***
%
%\usepackage{url}
% url.sty was written by Donald Arseneau. It provides better support for
% handling and breaking URLs. url.sty is already installed on most LaTeX
% systems. The latest version and documentation can be obtained at:
% http://www.ctan.org/pkg/url
% Basically, \url{my_url_here}.




% *** Do not adjust lengths that control margins, column widths, etc. ***
% *** Do not use packages that alter fonts (such as pslatex).         ***
% There should be no need to do such things with IEEEtran.cls V1.6 and later.
% (Unless specifically asked to do so by the journal or conference you plan
% to submit to, of course. )


% correct bad hyphenation here
\hyphenation{op-tical net-works semi-conduc-tor}


\begin{document}
	%
	% paper title
	% Titles are generally capitalized except for words such as a, an, and, as,
	% at, but, by, for, in, nor, of, on, or, the, to and up, which are usually
	% not capitalized unless they are the first or last word of the title.
	% Linebreaks \\ can be used within to get better formatting as desired.
	% Do not put math or special symbols in the title.
	\title{Efficient Human Inputs in Learning From Example}
	
	
	% author names and affiliations
	% use a multiple column layout for up to three different
	% affiliations








% TODO
	\author{\IEEEauthorblockN{Mohammad Nasirifar}
		\IEEEauthorblockA{Department of Computer Science\\
			University of Toronto\\
			farnasirim@cs.toronto.edu}
%		\and
%		\IEEEauthorblockN{Homer Simpson}
%		\IEEEauthorblockA{Twentieth Century Fox\\
%			Springfield, USA\\
%			Email: homer@thesimpsons.com}
%		\and
%		\IEEEauthorblockN{James Kirk\\ and Montgomery Scott}
%		\IEEEauthorblockA{Starfleet Academy\\
%			San Francisco, California 96678--2391\\
%			Telephone: (800) 555--1212\\
%			Fax: (888) 555--1212}
}
	
	
	
	
	
	
	
	
	
	
	
	
	
	
	
	% conference papers do not typically use \thanks and this command
	% is locked out in conference mode. If really needed, such as for
	% the acknowledgment of grants, issue a \IEEEoverridecommandlockouts
	% after \documentclass
	
	% for over three affiliations, or if they all won't fit within the width
	% of the page, use this alternative format:
	% 
	%\author{\IEEEauthorblockN{Michael Shell\IEEEauthorrefmark{1},
	%Homer Simpson\IEEEauthorrefmark{2},
	%James Kirk\IEEEauthorrefmark{3}, 
	%Montgomery Scott\IEEEauthorrefmark{3} and
	%Eldon Tyrell\IEEEauthorrefmark{4}}ci
	%\IEEEauthorblockA{\IEEEauthorrefmark{1}School of Electrical and Computer Engineering\\
	%Georgia Institute of Technology,
	%Atlanta, Georgia 30332--0250\\ Email: see http://www.michaelshell.org/contact.html}
	%\IEEEauthorblockA{\IEEEauthorrefmark{2}Twentieth Century Fox, Springfield, USA\\
	%Email: homer@thesimpsons.com}
	%\IEEEauthorblockA{\IEEEauthorrefmark{3}Starfleet Academy, San Francisco, California 96678-2391\\
	%Telephone: (800) 555--1212, Fax: (888) 555--1212}
	%\IEEEauthorblockA{\IEEEauthorrefmark{4}Tyrell Inc., 123 Replicant Street, Los Angeles, California 90210--4321}}
	
	
	
	
	% use for special paper notices
	%\IEEEspecialpapernotice{(Invited Paper)}
	
	
	
	
	% make the title area
	\maketitle
	
	% As a general rule, do not put math, special symbols or citations
	% in the abstract
	\begin{abstract}
		I tackle the problem of teaching specific expert level skills to learners. I leverage expert inputs
		for creating the learning material. The main contribution of this project is formulating this problem
		as a contextual bandit problem where different arms correspond to different hints/solutions for a
		particular learning task. Each task corresponds to a state and learner's previous skills define the
		context. Furthermore I provide a solution to a simplified case formulated in this way using 
		multi-armed bandits and Thompson sampling.
	\end{abstract}

	% no keywords
	
	% For peer review papers, you can put extra information on the cover
	% page as needed:
	% \ifCLASSOPTIONpeerreview
	% \begin{center} \bfseries EDICS Category: 3-BBND \end{center}
	% \fi
	%
	% For peerreview papers, this IEEEtran command inserts a page break and
	% creates the second title. It will be ignored for other modes.
	\IEEEpeerreviewmaketitle
	

	
	\section{Introduction}
	% no \IEEEPARstart
		Learning based on examples has been proved to be an effective way of improving one's abilities
	in a particular skill. This way of learning/teaching has been employed in chess education for a very long
	time: learners receive puzzles of real games played by grand-masters and are asked to come up
	with or rationalize the original move of the grand-master.\\
	This is widely employed in chess since the chess games are labeled with the correct approach by definition 
	and the rationale behind a move can be figured out by the learner or demonstrated to her by a teacher.\\
	There are however education scenarios in which the actions, labels, and/or explanations are not this clear.
	Take teaching programming as an example: there are lots of ways to accomplish a certain task, many of which
	are sub-optimal. In this project I have tried to create a framework to allow us to employ the aforementioned
	``puzzle-based learning'' in environments where obtaining puzzle explanations is more ``complicated'' than 
	the chess teaching scenario.
	

	\section{Related Work}
		The main motivation for this work comes from ``learnersourcing'' problem explanations idea described in 
		AXIS\cite{AXIS}, where the authors leveraged reflection and explanation by learners' about their own
		answers to both improve their learning performance, and furthermore, to generate and curate hints that
		can be used to help other learners understand a certain subject.\\
		The difference between this project and AXIS comes from the fact that in AXIS it is assumed that some
		students are capable of generating expert-grade hints for a subject they are studying. This can be true
		in more controlled environments such as academic ones, but can fall short in scenarios where more expertise
		is needed to create valuable hints. I believe that in these scenarios expert knowledge can be leveraged
		effectively to yield better results. This idea originates from the fact that we can imagine an extreme
		scenario in which we have access to one expert (professor) per learner (student), who would do the hint
		generation step instead of the student, as a consequence of which the hints will be of more quality.
		The obvious shortcoming in this approach however is that our resource of domain experts is quite sparse
		and we must be quite strict about the exact workload that we demand from them.\\
		
		This brings us to the next important background for this project that targets efficient human intervention
		in human in the loop systems \cite{HumanInTheLoop}. They model the human in the loop systems as Markov
		Decision Processes. The human (expert) resource constraint is resolved by making the human intervention
		process on-demand: a human will inform the system that she is willing to provide a new action for the MDP
		(create a new hint for some problem). That is, the platform moderator is free to use whatever amount of 
		resource that she decides to improves the system at any time. The system responds with a state that is 
		expected to yield the highest return if the human were to add the new action there. Under certain
		assumptions, a strategy is formulated to maximize the Expected Improvement (\textsc{EI}) over time.\\
		Unfortunately important hypotheses from this work depend upon the uncertainty in state changes imposed by
		the MDP, which is something that does not naturally fit the scenario of providing hints (actions) in
		different states (problems) since the core element of this problem is the uncertainty in the action rewards
		and the emphasis on the uncertainty in state changes is not very useful here. This is expected, as
		a general multi-armed bandit problem is equivalent to a single-state MDP.

	\section{Motivation}
	The main motivation for this project has been improving the process of training new programmers who have
	recently joined a company/project. This is a recurring problem in software engineering which has forced
	companies to allocate resources to developing their own training material. In this task the newcomers can
	hardly generate hints/explanations that is useful for others, as deep knowledge of different part of the
	project is crucial for this, regardless of the expertise of the new programmer in the fundamental
	programming skills. I will try to fit the to formulate this scenario using the model and it can be
	seen that no semantic dependency is formed between the two and other problems can be mapped to this 
	model conveniently by defining the required model specific parameters and metrics.
		
	\section{Idea and Modeling}
	
	We define different entities in the model:\\
	\begin{enumerate}
		\item Problem: An encapsulated unit that can be passed to a learner that she would try to solve. In the
		programming scenario, a problem consists of issue texts, tasks, requirement documents, and anything else
		which can define a unit of work for an already expert member of the team. The learner is then asked to
		try to carry out the task as the expert would. The answer to the problem is the patch that the learner
		creates by writing code. These ``problem'' are easy to generate in this scenario: teams store the
		history of their projects and the resources resulting in a tasking being done can be extracted from the
		team's issue tracker, documents, VCS, etc.
		
		\item Answer: An explanation of what the solution to a particular problem must ideally look like. In our
		scenario these can consist of the actual patch files and the documentation for the design decisions.
		These are generated by the experts to help the learners understand the underlying subjects of the problems.
		
		\item Learner Context: These are the context variables that define a certain learner. For programmers, this
		can correspond to their familiarity with different fields in software development.\\
	\end{enumerate}

	The human in the loop part of the system is modular and detachable. Without human inputs, the system functions
	like a normal system that approximates the solution to a (contextual) multi-armed bandit problem. This phase
	consists of choosing a hint for a learner visiting a particular problem, asking for her feedback about how
	helpful she things the hint has been, and updating the underlying parameters of the algorithm. We can operate
	any bandit algorithm for this part.\\
	
	The part that is of our interest however, is interacting with the expert to help her generate helpful hints.
	In this scenario, the expert must be given a state (problem) to generate a new hint for, in such way that
	maximizes the cumulative gain of the system over time.\\
	
	We assume that the learners will eventually go through all of the problems and will do so uniformly and 
	that the expert only asks to generate hints only when the state of the bandit algorithm is ``stable'',
	meaning different actions have had enough chance to be selected. This is to ensure that no particular
	problem would falsely overshadow other problems because they haven't had enough visitors, when we are
	choosing a target state for the expert. The first assumption is justifiable in the programming scenario as
	new programmers usually go through step by step tutorials which require them to complete a set of tasks
	one after the other. The second assumption will eventually be satisfied given the first assumption.
	Furthermore we take into consideration that multiple experts are involved in the hint generating process,
	each of whom possibly contributes to the solutions of a specific subset of the tasks. We assume the experts 
	to have different hint generating abilities.\\

	We define an expert $e$'s ability to generate hints as follows:
	$$
		Ability_e =  E[r\ |\ F_e] 
	$$
	Where r is the reward gained from presenting a hint to a learner, and
	F is the feedback data that we have received for this expert's hint:
	$$
		F_e = \{feedback \in hint\ |\ hint \in H_e\}
	$$
	$H_e$ being the hints that $e$ has created.
	Now based on the above definitions, we define the optimal problem choice for the new hint $H_{new}$ to be:
	$$
	p = \argmaxA_p \{ Pr(H_{new}\ is\ chosen\ from\ \{H_{new}\} \cup \{H_{i}\ |\ H_i \in p\})\}
	$$
	
	Where $H_{new}$ is dependent on $Ability_e$.\\
	
	To be more concrete, suppose that the experts' abilities and the hint rewards are distributed normally:
	$$
	r_e \sim {\rm N}(\mu_e,\sigma_e^2)
	$$
	for expert $e$, and
	$$
	r_h \sim {\rm N}(\mu_h,\sigma_h^2)
	$$
	for hint $h$.
	
	Therefore the answer the optimal choice for the problem that expert $e$ is going to generate hint for would be:
	$$
	p = \argmaxA_p \{ Pr(H_{new} > max \{H_{i}\ |\ H_i \in p\})\}
	$$
	We can consider each problem separately and solve for each probability value and get the maximum over all of them.
	In case of $|\{H_{i}\ |\ H_i \in p\}| = 1$ we are left with calculating $Pr(H_{new} > H_{old})$. 
	For two random variables distributed normally we have:
	$$
	X_1 \sim {\rm N}(\mu_1,\sigma_1^2), X_2 \sim {\rm N}(\mu_2,\sigma_2^2)
	$$
	$$
	{\rm P}(X_1  > X_2 ) = {\rm P}(X_1  - X_2  > 0) = 1 - {\rm P}(X_1  - X_2  \le 0).
	$$
	$$
	\mu := {\rm E}(X_1 - X_2) = \mu_1 - \mu_2
	$$
	$$
	\sigma^2 := {\rm Var}(X_1 - X_2) = \sigma_1^2 + \sigma_2^2.
	$$
	$$
	\frac{{X_1  - X_2  - \mu}}{{\sigma}} \sim {\rm N}(0,1),
	$$
	Therefore the problem is reduced to calculating the probability of one random variable being bigger than a particular value.
	Also in $Pr(H_{new} > max \{H_{i}\ |\ H_i \in p\})$ every $H_i$ can be considered independently from others, allowing us to 
	extend the two variables case to solve the original problem when $|\{H_{i}\ |\ H_i \in p\}| > 1$.


	\section{Implementation}		
	For simplicity, I have only taken into account a boolean positive/negative feedback to a provided hint and therefore have used Beta distribution to model the rewards of the hints and agents. Furthermore I have considered only a single expert in the system. For 
	calculating the optimal problem based on the above policy, I have used the less efficient Markov Chain Monte Carlo instead of the
	closed form.
	
	\section{Evaluation}
	% no \IEEEPARstart
	To be able to test the approach on a more accessible audience, I have opted to using algorithmic problems instead of programming
	tasks. I provide the learner with an an algorithmic problem that I ask them to think about. After they do they would request for
	a hint. The hint will be selected using Thompson sampling from the current hints in the system. After that they provide a boolean
	feedback to signify whether or not they think the hint has been useful. The expert can query the system to ask which problem would
	make the most impactful choice if she were to generate a hint, and after the response, the expert can add the new hint to the problem.
	
	
	\section{Shortcomings and Future Work}
	I have not taken into account what happens when we have create new problems in the analysis. Intuitively, the semantics of the model
	make sense and I would expect everything to work fine in that case too, but I have not gone through any formal analysis.\\
	I think the experts' ability can be captured more strongly. It looks like that there is a lot of ``linearization'' happening when
	we combine the whole history of the feedbacks to an experts' hints together and we should be able to do better to ``lose track of''
	less data in the process. One approach that naturally comes to mind in the bandits and Bayesian inference context is modeling the
	ability function as a probability distribution of probabilities.

% TODO
%	This demo file is intended to serve as a ``starter file''
%	for IEEE conference papers produced under \LaTeX\ using
%	IEEEtran.cls version 1.8b and later.
%	% You must have at least 2 lines in the paragraph with the drop letter
%	% (should never be an issue)
%	I wish you the best of success.
	
%	\hfill mds
%	
%	\hfill August 26, 2015
	
%	\subsection{Subs\texttt{ection Heading Here}
%	Subsection text here.
%	
%	
%	\subsubsection{Subsubsection Heading Here}
%	Subsubsection tex}t here.
	
	
	% An example of a floating figure using the graphicx package.
	% Note that \label must occur AFTER (or within) \caption.
	% For figures, \caption should occur after the \includegraphics.
	% Note that IEEEtran v1.7 and later has special internal code that
	% is designed to preserve the operation of \label within \caption
	% even when the captionsoff option is in effect. However, because
	% of issues like this, it may be the safest practice to put all your
	% \label just after \caption rather than within \caption{}.
	%
	% Reminder: the "draftcls" or "draftclsnofoot", not "draft", class
	% option should be used if it is desired that the figures are to be
	% displayed while in draft mode.
	%
	%\begin{figure}[!t]
	%\centering
	%\includegraphics[width=2.5in]{myfigure}
	% where an .eps filename suffix will be assumed under latex, 
	% and a .pdf suffix will be assumed for pdflatex; or what has been declared
	% via \DeclareGraphicsExtensions.
	%\caption{Simulation results for the network.}
	%\label{fig_sim}
	%\end{figure}
	
	% Note that the IEEE typically puts floats only at the top, even when this
	% results in a large percentage of a column being occupied by floats.
	
	
	% An example of a double column floating figure using two subfigures.
	% (The subfig.sty package must be loaded for this to work.)
	% The subfigure \label commands are set within each subfloat command,
	% and the \label for the overall figure must come after \caption.
	% \hfil is used as a separator to get equal spacing.
	% Watch out that the combined width of all the subfigures on a 
	% line do not exceed the text width or a line break will occur.
	%
	%\begin{figure*}[!t]
	%\centering
	%\subfloat[Case I]{\includegraphics[width=2.5in]{box}%
	%\label{fig_first_case}}
	%\hfil
	%\subfloat[Case II]{\includegraphics[width=2.5in]{box}%
	%\label{fig_second_case}}
	%\caption{Simulation results for the network.}
	%\label{fig_sim}
	%\end{figure*}
	%
	% Note that often IEEE papers with subfigures do not employ subfigure
	% captions (using the optional argument to \subfloat[]), but instead will
	% reference/describe all of them (a), (b), etc., within the main caption.
	% Be aware that for subfig.sty to generate the (a), (b), etc., subfigure
	% labels, the optional argument to \subfloat must be present. If a
	% subcaption is not desired, just leave its contents blank,
	% e.g., \subfloat[].
	
	
	% An example of a floating table. Note that, for IEEE style tables, the
	% \caption command should come BEFORE the table and, given that table
	% captions serve much like titles, are usually capitalized except for words
	% such as a, an, and, as, at, but, by, for, in, nor, of, on, or, the, to
	% and up, which are usually not capitalized unless they are the first or
	% last word of the caption. Table text will default to \footnotesize as
	% the IEEE normally uses this smaller font for tables.
	% The \label must come after \caption as always.
	%
	%\begin{table}[!t]
	%% increase table row spacing, adjust to taste
	%\renewcommand{\arraystretch}{1.3}
	% if using array.sty, it might be a good idea to tweak the value of
	% \extrarowheight as needed to properly center the text within the cells
	%\caption{An Example of a Table}
	%\label{table_example}
	%\centering
	%% Some packages, such as MDW tools, offer better commands for making tables
	%% than the plain LaTeX2e tabular which is used here.
	%\begin{tabular}{|c||c|}
	%\hline
	%One & Two\\
	%\hline
	%Three & Four\\
	%\hline
	%\end{tabular}
	%\end{table}
	
	
	% Note that the IEEE does not put floats in the very first column
	% - or typically anywhere on the first page for that matter. Also,
	% in-text middle ("here") positioning is typically not used, but it
	% is allowed and encouraged for Computer Society conferences (but
	% not Computer Society journals). Most IEEE journals/conferences use
	% top floats exclusively. 
	% Note that, LaTeX2e, unlike IEEE journals/conferences, places
	% footnotes above bottom floats. This can be corrected via the
	% \fnbelowfloat command of the stfloats package.
	
	
	
	
%	\section{Conclusion}
%	The conclusion goes here.
	
	
	
	
	% conference papers do not normally have an appendix
	
	
	% use section* for acknowledgment
	% \section*{Acknowledgment}
	% TODO
	
	
	
	
	% trigger a \newpage just before the given reference
	% number - used to balance the columns on the last page
	% adjust value as needed - may need to be readjusted if
	% the document is modified later
	%\IEEEtriggeratref{8}
	% The "triggered" command can be changed if desired:
	%\IEEEtriggercmd{\enlargethispage{-5in}}
	% references section
	
	% can use a bibliography generated by BibTeX as a .bbl file
	% BibTeX documentation can be easily obtained at:
	% http://mirror.ctan.org/biblio/bibtex/contrib/doc/
	% The IEEEtran BibTeX style support page is at:
	% http://www.michaelshell.org/tex/ieeetran/bibtex/
	\bibliographystyle{IEEEtran}
	% argument is your BibTeX string definitions and bibliography database(s)


	\bibliography{iai}

	%
	% <OR> manually copy in the resultant .bbl file
	% set second argument of \begin to the number of references
	% (used to reserve space for the reference number labels box)
%	\begin{thebibliography}{1}
%		
%		\bibitem{IEEEhowto:kopka}
%		H.~Kopka and P.~W. Daly, \emph{A Guide to \LaTeX}, 3rd~ed.\hskip 1em plus
%		0.5em minus 0.4em\relax Harlow, England: Addison-Wesley, 1999.
%		
%	\end{thebibliography}
	%TODO
	
	
	
	% that's all folks
\end{document}


